\newcommand{\imprimirerrata}{%
    \begin{errata}
        Elemento opcional. Exemplo:
        
        \vspace{\onelineskip}
        
        FERRIGNO, C. R. A. \textbf{Tratamento de neoplasias ósseas apendiculares com
        reimplantação de enxerto ósseo autólogo autoclavado associado ao plasma
        rico em plaquetas}: estudo crítico na cirurgia de preservação de membro em
        cães. 2011. 128 f. Tese (Livre-Docência) - Faculdade de Medicina Veterinária e
        Zootecnia, Universidade de São Paulo, São Paulo, 2011.
        
        \begin{table}[htb]
            \center
            \footnotesize
            \begin{tabular}{|p{1.4cm}|p{1cm}|p{3cm}|p{3cm}|}
                \hline
                \textbf{Folha} & \textbf{Linha}  & \textbf{Onde se lê}  & \textbf{Leia-se}  \\
                \hline
                1 & 10 & auto-conclavo & autoconclavo\\
                \hline
            \end{tabular}
        \end{table}
    
    \end{errata}
}

\newcommand{\imprimirdedicatoria}{%
    \begin{dedicatoria}
        \vspace*{\fill}
        \centering
        \noindent
        \textit{ Este trabalho é dedicado às crianças adultas que,\\
        quando pequenas, sonharam em se tornar cientistas.} \vspace*{\fill}
    \end{dedicatoria}
}

\newcommand{\imprimiragradecimentos}{%
    \begin{agradecimentos}
        Os agradecimentos principais são direcionados à Gerald Weber, Miguel Frasson,
        Leslie H. Watter, Bruno Parente Lima, Flávio de Vasconcellos Corrêa, Otavio Real
        Salvador, Renato Machnievscz\footnote{Os nomes dos integrantes do primeiro
        projeto abn\TeX\ foram extraídos de
        \url{http://codigolivre.org.br/projects/abntex/}} e todos aqueles que
        contribuíram para que a produção de trabalhos acadêmicos conforme
        as normas ABNT com \LaTeX\ fosse possível.
        
        Agradecimentos especiais são direcionados ao Centro de Pesquisa em Arquitetura
        da Informação\footnote{\url{http://www.cpai.unb.br/}} da Universidade de
        Brasília (CPAI), ao grupo de usuários
        \emph{latex-br}\footnote{\url{http://groups.google.com/group/latex-br}} e aos
        novos voluntários do grupo
        \emph{\abnTeX}\footnote{\url{http://groups.google.com/group/abntex2} e
        \url{http://www.abntex.net.br/}}~que contribuíram e que ainda
        contribuirão para a evolução do \abnTeX.    
    \end{agradecimentos}
}

\newcommand{\imprimirepigrafe}{%
    \begin{epigrafe}
        \vspace*{\fill}
        \begin{flushright}
            \textit{``Não vos amoldeis às estruturas deste mundo, \\
            mas transformai-vos pela renovação da mente, \\
            a fim de distinguir qual é a vontade de Deus: \\
            o que é bom, o que Lhe é agradável, o que é perfeito.\\
            (Bíblia Sagrada, Romanos 12, 2)}
        \end{flushright}
    \end{epigrafe}
}

\newcommand{\imprimirresumo}{%
    \begin{resumo}
        Segundo as normas da ABNT, o resumo deve ressaltar o objetivo, o método,
        os resultados e as conclusões do documento. A ordem e a extensão destes
        itens dependem do tipo de resumo (informativo ou indicativo) e do
        tratamento que cada item recebe no documento original. O resumo deve ser
        precedido da referência do documento, com exceção do resumo inserido no
        próprio documento. (\ldots) As palavras-chave devem figurar logo abaixo do
        resumo, antecedidas da expressão Palavras-chave:, separadas entre si por
        ponto e finalizadas também por ponto.
       
        \textbf{Palavras-chave}: latex. abntex. editoração de texto.
    \end{resumo}
}
\newcommand{\imprimirabstract}{% Inglês
    \begin{otherlanguage}{english}
        \begin{resumo}
            This is the english abstract.

            \textbf{Keywords}: latex. abntex. text editoration.
        \end{resumo}
    \end{otherlanguage}
}
\newcommand{\imprimirresume}{% Francês
    \begin{otherlanguage}{french}
        \begin{resumo}[Résumé]
            Il s'agit d'un résumé en français.
         
           \textbf{Mots-clés}: latex. abntex. publication de textes.
        \end{resumo}
    \end{otherlanguage}
}
\newcommand{\imprimirresumen}{% Espanhol
    \begin{otherlanguage}{spanish}
        \begin{resumo}[Resumen]
            Este es el resumen en español.
           
            \textbf{Palabras clave}: latex. abntex. publicación de textos.
        \end{resumo}
    \end{otherlanguage}
} 

\newcommand{\imprimirlistadesiglas}{%
    \begin{siglas}
        \item[ABNT] Associação Brasileira de Normas Técnicas
        \item[abnTeX] ABsurdas Normas para TeX
    \end{siglas}
}
\newcommand{\imprimirlistadesimbolos}{%
    \begin{simbolos}
        \item[$ \Gamma $] Letra grega Gama
        \item[$ \Lambda $] Lambda
        \item[$ \zeta $] Letra grega minúscula zeta
        \item[$ \in $] Pertence
    \end{simbolos}
}

\newcommand{\imprimirapendices}{%
    \begin{apendicesenv}
        \partapendices  % imprime uma página indicando o início dos apêndices

        \chapter{Apêndice 1}
            \lipsum
        
        \chapter{Apêndice 2}
            \lipsum
    \end{apendicesenv}
}

\newcommand{\imprimiranexos}{%
    \begin{anexosenv}
        \partanexos     % imprime uma página indicando o início dos anexos

        \chapter{Anexo 1}
            \lipsum
        
        \chapter{Anexo 2}
            \lipsum
    \end{anexosenv}
}
